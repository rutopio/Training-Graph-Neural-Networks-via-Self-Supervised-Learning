% !TeX root = ../main.tex

\begin{acknowledgement}

研究所的日子實在過得飛快,當初排隊領新學生證的回憶彷彿還歷歷在目,卻又在轉瞬間迎接這個即將與學校告別的仲夏。本篇論文得以順利地付梓成冊,一路上真的承蒙很多人的幫忙與鼓勵。在這裡,我想試著用有限的頁數和粗淺的文筆,向這些人表達我無限的感激和敬意。

首先,我要感謝我的指導教授顏佐榕老師。顏老師給予了我在時間規劃上充足的自由,並適時地提供我研究方向的建議與論文寫作的引導,讓我能依照自己的步調,很彈性地嘗試自己想挑戰的事情、摸索自己有興趣的議題。此外,顏老師在日常中的關懷備至、以及每次 meeting 後的閒話家常,也常常讓我覺得自己很幸運能成為您的學生,在此向您致上誠摯的謝意。

感謝我的共同指導教授沈俊嚴老師、口試委員杜憶萍老師、黃冠華老師以及黃信誠老師。撰寫論文是一趟孤獨的長途旅程,時常會有自己沒注意到的盲點或是問題存在,正所謂當局者迷、旁觀者清,若旁人能夠適時的給予提點或建議,將使文章的內容臻於完善,謝謝以上老師們對這篇論文的指教與討論。

再者,也要珍惜這幾年一路陪伴我的好友們,時而分享笑點自娛娛人,時而承擔我壓力的緩衝劑。謝謝吳調亭霖、治翔法師、張檢子儀、前和解律師每週的言論自由日,一起吃壽司、一起踩點、一起擺拍、一起唱K、一起製圖、一起討拍,為煢煢的研究生活帶來不少歡樂,也讓我順利地帶走了兩隻大壽司抱枕,祝福未來大家的人生旅途都能充滿刺激。

謝謝遠在日本的李公亦修三不五時捎來奇葩點子,雖然有時候真的領先人類文明太多年,但每每都能讓我嘆為觀止、增廣見聞;謝謝 Iris 在職涯與藝術方面的啟發與討論,偶爾帶給我迷一般的自信心,或多或少影響了我的眼界和思考模式;謝謝 Hina 在我遇到各種變故的那些日子裡,成為我可靠的傾訴對象,把我從迷航的邊緣救了回來;謝謝芊妤、新愷、湘斌的日常玩笑與取暖,雖然因為疫情的關係有點久沒揪小火鍋了,但與你們的小窗窗總能為平淡如水的研究之路帶來一點漣漪。

B05 與 R09 的同窗好友們,力仁、湛然、敦敏、偉樂、知遙、瀞瑩、禹翔、佳穎、小帥、季祐等,能與你們一起討論那些詭譎到不行的題目和解法、挑戰一道又一道的 baseline,互相切磋學術與分享新鮮事,真的是很讓人愉快的時刻。當中有些人即將邁入職場、也有人選擇繼續攻讀學位,在這邊祝福大家都能順順利利,成為自己理想中的那個人。

最後,我要特別感謝家人與長輩們的包容和鼓勵。感謝妹妹在日常中給予的關心和溫暖,容忍我三不五時的抱怨和聒噪;也感謝老爸撐起一個家,常常和我從天南聊到地北,不定期提供一些神奇的觀點供我參考;感謝阿母一直以來的愛與付出,辛辛苦苦提拔我跟妹妹長大,我們都會把自己照顧得很好;感謝阿公與阿嬤們的早安貼圖,還總會準備好大餐等我們回老家;感謝乾媽、阿姨、舅舅、姑姑、堂表親們從小到大的體貼,每每過節時總會追蹤我論文的進度;感謝淑樺阿姨和阿嬌老師從我還是小屁孩的時候便一路關心我到大,長大後也會不時詢問我研究所的近況;感謝純婉老師在高中時給予我的啟蒙,培養我那顆勇於挑戰權威的企圖心。因為有你們的細心呵護,才使我有機會踏上研究之路,讓我可以在知識的追求上得以心無旁鶩。

從老家大甲、故鄉龍潭、再到定居台北;從小大一初入文學院、在社科院披上紫色領巾、再到現在即將離開電資學院,我常覺得我的腦袋裡住著一個矛盾的靈魂,讓我的想法時而協和、時而卻又衝突——就像個徘徊於科技與人文十字路口的浪人,在理性和浪漫的光譜之間擺盪著。

幸運的是,我能遇到那麼多願意相信我的人,讓這些點點滴滴的想法成為我茁壯的養分,一步一步編織成為有趣的點子,最終得以完成這篇論文。僅以這份拙作作為我十八年學生身份的終章,並期許在往後的日子裡莫忘初衷,繼續探索自己無窮的可能性。

O ever youthful, O ever weeping.

\rightline{何青儒~~謹誌}
\rightline{2022年6月10日~~於國立臺灣大學}


\end{acknowledgement}